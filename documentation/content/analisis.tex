\section{Analyse}
In den folgenden Kapiteln, werden die punkte, welche bereits in der Themeneingabe
definiert wurden, präzisiert.

\subsection{Ausgangslage}
Um, als Indieentwickler, Projekte zu plaanen, hat man heutzutage bereits eine relativ grosse Auswahl an
Projektmanagement Tools.
Tools wie Azure DevOps sind jedoch hauptsächlich für grössere Teams gedacht und der mehraufwand schreckt die meisten Einzelentwickler ab.
Desshalb wird meisst auf ein Projektmanagement tool verzichtet und man behilft sich mit Post-Its oder einem Whiteboard.\\
Die meisten Projekt management Tools haben einen ähnlichen aufbau, welcher sich auch bewährt hat.
Um ein Produkt in dieser Kategorie zu entwickeln, ist es sicher eine gute Idee, bestehende Produkte etwas genauer anzusehen und vor und nachteile 
im bezug auf Indieentwickler rauszufiltern.

\subsubsection{Analyse von bestehenden Lösungen}


\begin{figure}[H]
    \begin{center}
        \includegraphics[width=5cm]{../content/images/Trello/TrelloLogo.png}
        \caption{TRELLO Logo}
    \end{center}
\end{figure}

Trello ist ein Online-Tool zum verwalten von Aufgaben und gehört dem Unternehmen Atlassan. 
Das beliebte Online-Tool ist seit 2011 auf dem Markt und erfeut sich an milionen von Benutzern.


%------------------------------------------------------- 
%TRELLO BOARD SCREENSHOT
\begin{figure}[H]
    \begin{center}
        \includegraphics[width=16cm]{../content/images/Trello/TrelloBoard.png}
        \caption{TRELLO board}
    \end{center}
\end{figure}
%------------------------------------------------------- 

Das Layout entspricht einem klassischen Kanban-Board, welches jedoch nach belieben angepasst und
modifiziert werden kann. Das geschäftsmodell von trello lässt sich am ehsten als ''freemium'' bezeichen, die nötigsten Funktionen
sind gratis, reicht dies nicht aus, so kann ein Premium-Abo für 10.- pro monat abgeschlossen werden.
Beim Testen des Dienstes ist mir das penetrante anbieten des 30Days-Free-Trial angebotes, welche sich nach abschluss der 30 Tage automatisch verlängert
besonders negativ aufgefallen.\\
\space
%------------------------------------------------------- 
%TRELLO CARD SCREENSHOT WRAPPER
\begin{figure}[H]
    \begin{center}
        \includegraphics[width=5cm]{../content/images/Trello/TrelloCard.png}
        \caption{TRELLO Card}
    \end{center}
\end{figure}
%------------------------------------------------------- 

Dem Trello-Board können sogenannte ''Cards'' hinzugefügt werden. Den ''Cards'' kann man standartmässig
einen Titel, eine Beschreibung und Kommentare hinzufügen. Es lassen sich ausserdem
Personen, Labels, Checklists, Covers und Anhänge anfügen.\\
Fängt man das erste Projekt mit Trello an, so scheinen die Funktionen relativ übersichtlich zu sein.
Die grösste stärke, meiner Meinung nach, liegt jedoch in der konfigurierbarkeit von Trello. Es können nähmlich
''PowerUps'' eingefügt werden. PowerUps sind Plugins, mit denen die Funktionen von Trello beliebig erweitert werden können.
Zum Beispiel gibt es standartmässig keine möglichkeit den Cards irgendpwelche StoryPoints oder Stunden zuzuweisen.
Fügt man jedoch das entsprechende PowerUp hinzu so ist dies kein Problem mehr.\\
Und es gibt fast für alles ein entsprechendes PowerUp, so können auch Excel-Daten mittels PowerUp importiert werden.
Dem Benutzer sind also nur wehnig Grenzen gesetzt, was die Personalisierung von Trello angeht.\\

%------------------------------------------------------- 
%TRELLO POWERUPS SCREENSHOT WRAPPER
\begin{figure}[H]
    \begin{center}
        \includegraphics[width=8cm]{../content/images/Trello/PowerUps.png}
        \caption{TRELLO PowerUps}
    \end{center}
\end{figure}
%------------------------------------------------------- 

PowerUps können in einer art App-Store durchsucht und hinzugefügt werden.\\
Da die PowerUps auch von Trello-Usern erstellt werden können, gibt es fast für jeden Use-Case ein entsprechendes PowerUp.
 
%------------------------------------------------------- 
%TRELLO DETAILS TABLE
\begin{table}[H]
    \centering
    \settowidth\tymin{executeIncomingCommand()}
    \setlength\extrarowheight{2pt}
    \begin{tabulary}{1.0\textwidth}{|L|L|}
      \hline
      \textbf{Besitzer} &
      Atlassan\\
      \hline
      \textbf{Gründung} &
      2011\\
      \hline
      \textbf{Plattform} &
      Web\\
      \hline
      \textbf{Layout} &
      Kanban\\
      \hline
      \textbf{Geschäftsmodell} &
      Freemium\\
      \hline
    \end{tabulary}
    \caption{TRELLO Details}
  \end{table}
%------------------------------------------------------- 

%------------------------------------------------------- 
%TRELLO RATING TABLE
\begin{table}[H]
    \centering
    \settowidth\tymin{executeIncomingCommand()}
    \setlength\extrarowheight{2pt}
    \begin{tabulary}{1.0\textwidth}{|L|L|L|}
      \hline
      \textbf{Bewertungspunkt} &
      \textbf{Bewertung} &
      \textbf{Begründung} \\
      \hline
      \textbf{Benutzerfreundlichkeit} &
      ****&
      Die Hohe konfigurierbarkeit führt zwangsläufig dazu, dass die Übersichtlichkeit der Funktionen etwas leidet\\
      \hline
      \textbf{Darstellung} &
      *****&
      Klassisches Kanbanboard, Backgrounds und Design können nach belieben angepasst werden\\
      \hline
      \textbf{Usability} &
      *****&
      Da Trello eine Webapplikation ist, ist sie auf allen webfähigen Geräten zugänglich\\
      \hline
      \textbf{Funktionalität} &
      *****&
      Durch die PowerUps kann die Funktionalität nach belieben erweitert werden\\
      \hline
      \textbf{Preis} &
      ****&
      Die Gratisversion ist absolut ausreichend, jedoch wird man immer wieder dazu gedrängt ein Premium-Abo abzuschliessen welches nicht gerade günstig ist\\
      \hline
    \end{tabulary}
    \caption{TRELLO Bewertung}
  \end{table}
%------------------------------------------------------- 
\space
\textbf{Punkte zur berücksichtigung in eigenem Projekt:}
\begin{itemize}
    \item Konfigurierbarkeit mittels anfügbarer Komponenten (PowerUps)
    \item Kanban Layout
    \item Design
\end{itemize}

  
  
\newpage

\begin{figure}[H]
    \begin{center}
        \includegraphics[width=5cm]{../content/images/monday.com/MondayLogo.png}
        \caption{Monday.com Logo}
    \end{center}
\end{figure}

Monday.com ist eine Online-Plattform zum erstellen von Anwendungen und Arbeitsverwaltungs Software.
Die Plattform wurde 2014 veröffentlicht.

%------------------------------------------------------- 
%TRELLO BOARD SCREENSHOT
\begin{figure}[H]
    \begin{center}
        \includegraphics[width=16cm]{../content/images/monday.com/MondayBoard.png}
        \caption{TRELLO board}
    \end{center}
\end{figure}
%------------------------------------------------------- 

In der Standartkonfiguration werden Tasks aufgelistet und nicht als Kanbanboard dargestellt.
Wird ein neuer Workspace erstellt, so klickt man sich erst durch eine Reihe fragen, deren Resultate anschliessend das Layout und die Funktionen
des Workspaces vordefinieren.\\
Die Ansicht des Workspaces ist nach belieben konfigurierbar, so können  Tasks nach Monaten gruppiert werden oder aber auch zB. nach Sprints oder Thema.
Monday.com setzt ebenfalls auf das Freemium Geschäftsmodell, für ein Abo wird jedoch nicht so penetrant geworben wie bei anderen Anbietern.\\
\space
%------------------------------------------------------- 
%TRELLO CARD SCREENSHOT WRAPPER
\begin{wrapfigure}{l}{0.4\textwidth}
    \begin{center}
        \includegraphics[width=5cm]{../content/images/monday.com/MondayTask.png}
        \caption{Monday.com Task}
    \end{center}
\end{wrapfigure}
%------------------------------------------------------- 

Monday.com können Tasks erstellt werden, im gegensatz zu anderen Anbietern sind diese Tasks
sehr minimalistisch gehalten und enthalten lediglich einen Titel und eine Beschreibung.\\
Den Tasks können jedoch weitere Views angefügt werden, in deren weiter felder und Funktionen zu verfügung stehen.
Vom Prinzip her gleicht es zwar den PowerUps von trello, dort finde ich jedoch die Anbindung um einiges
intuitiver und effizienter.\\

%------------------------------------------------------- 
%TRELLO POWERUPS SCREENSHOT WRAPPER
\begin{figure}[H]
    \begin{center}
        \includegraphics[width=8cm]{../content/images/monday.com/MondayItemViewsCenter.png}
        \caption{TRELLO PowerUps}
    \end{center}
\end{figure}
%------------------------------------------------------- 

Views können im Item-View-Center durchstöbert und hinzugefügt werfden. 
Die hinzugefügten Views erscheinen dann in einem separaten Tab im Task. In den Views ist es ausserdem möglich Widgets einzufügen
und das generelle Layout zu konfigurieren.\\
Im Item-View-Center findet man native Monday.com Views aber auch ausreichend andere Views von drittanbietern.

%------------------------------------------------------- 
%TRELLO DETAILS TABLE
\begin{table}[H]
    \centering
    \settowidth\tymin{executeIncomingCommand()}
    \setlength\extrarowheight{2pt}
    \begin{tabulary}{1.0\textwidth}{|L|L|}
      \hline
      \textbf{Besitzer} &
      monday.com\\
      \hline
      \textbf{Gründung} &
      2012\\
      \hline
      \textbf{Plattform} &
      Web\\
      \hline
      \textbf{Layout} &
      Liste\\
      \hline
      \textbf{Geschäftsmodell} &
      Freemium\\
      \hline
    \end{tabulary}
    \caption{Monday.com Details}
  \end{table}
%------------------------------------------------------- 

%------------------------------------------------------- 
%TRELLO RATING TABLE
\begin{table}[H]
    \centering
    \settowidth\tymin{executeIncomingCommand()}
    \setlength\extrarowheight{2pt}
    \begin{tabulary}{1.0\textwidth}{|L|L|L|}
      \hline
      \textbf{Bewertungspunkt} &
      \textbf{Bewertung} &
      \textbf{Begründung} \\
      \hline
      \textbf{Benutzerfreundlichkeit} &
      ***&
      --\\
      \hline
      \textbf{Darstellung} &
      ****&
     Listenansichten werden schnell unübersichtlich\\
      \hline
      \textbf{Usability} &
      *****&
      Da monday.com eine Webapplikation ist, ist sie auf allen webfähigen Geräten zugänglich + gut optimiert für Mobilgeräte\\
      \hline
      \textbf{Funktionalität} &
      ****&
      Durch Views können Funktionalitäten hinzugefügt werden jedoch ist die auswahl deutlich kleiner als z.B. bei Trello\\
      \hline
      \textbf{Preis} &
      *****&
      Gratis version reicht vollkommen\\
      \hline
    \end{tabulary}
    \caption{TRELLO Bewertung}
  \end{table}
%------------------------------------------------------- 
\space
\textbf{Punkte zur berücksichtigung in eigenem Projekt:}
\begin{itemize}
    \item Lieber keine Listenansicht verwenden
\end{itemize}

  
  
\newpage

\subsection{Stakeholder}

Im folgenden abschnitt werden die wichtigsten Stakeholder ermittelt.
Dabei werden deren Interessen an das Projekt analysiert und daraus eine entsprechende Gewichtung vorgenommen.
Die erkenntnisse aus der Stakeholderanalyse fliessen anschliessend in die
Definition der Ziele und Anforderungen ein.\\
Folgende Stakeholder werden berücksichtigt:\\

\begin{itemize}
    \item TEKO
    \item Indieentwickler
    \item Konsumenten von Indiespielen
    \item Ich
    \item GitHub
    \item Unity
\end{itemize}
\newpage

\subsubsection{Stakeholdergewichtung}
\vspace{1cm}
\begin{figure}[H]
    \centering
\rotatebox{90}{\begin{minipage}{0.75\textheight}
    \includegraphics[width=20cm]{../content/excel/StakeholderAnalysis.png}
\end{minipage}}
\caption{Stakeholder Analyse}
\end{figure}
\newpage

\begin{center}
    
\begin{tikzpicture}
    \pie[text=legend, sum=auto, explode=0.1, radius = 4]{
        11/Ich, 
        4/Indieentwickler, 
        10/Teko, 
        3/Videospielkonumenten, 
        4/Unity Technologies, 
        3/GitHub Inc.
        }
\end{tikzpicture}
\end{center}




\newpage
\subsection{Ziele}
\begin{itemize}
    \item Projektmanagement und Projekt soll als einheit abgespeichert werden können
    \item Schnellere erfassung von Tasks während der Entwicklungsphase
    \item Höhere Rate von abgeschlossenen Projekten
    \item Höhere Qualität von abgeschlossenen Projekten
\end{itemize}
\newpage
\subsection{Anforderungen}
\subsubsection{Applikation}
\begin{itemize}
    \item Unity-Projekte können in der Applikation geplant werden
    \item Aktivitäten können in der Applikation erstellt, bearbeitet und gelöscht werden
    \item Aktivitäten können terminiert werden (Start -und Enddatum)
    \item Aktivitäten können kommentiert werden
    \item Aktivitäten können priorisiert werden
    \item Aktivitäten können punkte zugewiesen werden (z.B. Story Points oder Stunden)
    \item Aktivitäten können gruppiert werden (z.B. zu Phasen)
    \item Gruppen von Aktivitäten können kommentiert werden*
    \item Gruppen können terminiert werden (Start- und Enddatum)
    \item Gruppen können priorisiert werden
    \item Gruppen können punkte zugewiesen werden (z.B. Story Points oder Stunden)
    \item Bestehende Unity-Projekte können der Applikation hinzugefügt werden
    \item Meilensteine können in der Applikation geplant werden
    \item Aktivitäten können Benutzern zugewiesen werden
    \item Wird ein Unity-Projekt der Applikation hinzugefügt, so wird das entsprechende Custom-Editor-Script dem Projektordner hinzugefüg
\end{itemize}
\subsubsection{In Unity integriertes Benutzerinterface}
\begin{itemize}
    \item  Aktivitäten zum aktuell geöffneten Projekt können erstellt werden
    \item Aktivitäten zum aktuell geöffneten Projekt können bearbeitet werden
    \item Projektinformationen zum aktuell geöffneten Projekt (Beschreibung, Meilensteine usw.) können angesehen werden
\end{itemize}
\subsubsection{Nicht funktionale Anforderungen}
\begin{itemize}
    \item Projektdaten werden beim Projekt selbst hinterlegt
    \item Einfache erweiterbarkeit der Features
    \item Projektdaten können mitels versionsverwaltung (z.B. GIT) synchronisiert werden
\end{itemize}
\newpage
\subsection{Abgrenzung}
\begin{itemize}
    \item Projektdaten (Aktivitäten, Gruppen usw.) werden nicht auf einer Datenbank persistiert
    \item Die Softwarelösung ist ausschliesslich auf Windows Geräten verfügbar
    \item Synchronisation der Daten z.B. über einen Webserver ist nicht möglich
\end{itemize}
\newpage
\subsection{Risikomanagement}
\subsubsection{Risikoerkennung}
%------------------------------------------------------- 
%RISK EVALUATION TABLE
\begin{table}[H]
    \centering
    \settowidth\tymin{executeIncomingCommand()}
    \setlength\extrarowheight{2pt}
    \begin{tabulary}{1.0\textwidth}{| m{3cm} | m{9cm} |}
      \hline
      \textbf{Risiko-Art} &
      \textbf{Mögliche Risiken}\\
      \hline
      \textbf{Technologisch} &
        \begin{itemize}
            \item Das Speichern der Projektdaten im Projektordner ist zu wehnig performant 
        \end{itemize}\\
      \hline
      \textbf{Personell} &
      \begin{itemize}
        \item Krankheitsausfall
        \end{itemize}\\
      \hline
      \textbf{Werkzeuge} &
      \begin{itemize}
        \item Änderungen durch Unity in Form von Updates, könnten den Einsatz von Custom Editor Windows erschweren oder verunmöglichen
        \end{itemize}\\
      \hline
      \textbf{Anforderung} &
        \begin{itemize}
        \item Nachträgliche Änderungen / Erweiterungen werden verlangt
        \end{itemize}\\
      \hline
      \textbf{Schätzung} &
      \begin{itemize}
        \item Die Entwicklungsdauer wurde unterschätzt
        \item Die Dauer der Dokumentationen wurde unterschätzt
      \end{itemize}\\
      \hline
    \end{tabulary}
    \caption{Risikoerkennung}
  \end{table}
%------------------------------------------------------- 
\subsubsection{Risikogewichtung}
Nachfolgend werden die erkannten Risiken nach ihrer Eintrittswahrscheinlichkeit (0-5) und den 
möglichen Auswirkungen (0-5) bewertet.\\
%------------------------------------------------------- 
%RISK WEIGHT TABLE
\begin{table}[H]
    \centering
    \settowidth\tymin{executeIncomingCommand()}
    \setlength\extrarowheight{2pt}
    \begin{tabulary}{1.0\textwidth}{|L|L|L|}
      \hline
      \textbf{Risiko} &
      \textbf{Eintrittswarscheinlichkeit}&
      \textbf{Auswirkung}\\
      \hline
      Das Speichern der Projektdaten im Projektordner ist zu wehnig performant &
      3 &
      2\\
      \hline
      Krankheitsausfall &
      3 &
      2\\
      \hline
      Änderungen durch Unity in Form von Updates, könnten den Einsatz von Custom Editor Windows erschweren oder verunmöglichen &
      1 &
      4\\
      \hline
      Nachträgliche Änderungen / Erweiterungen werden verlangt &
      2 &
      3\\
      \hline
      Die Entwicklungsdauer wurde unterschätzt &
      3 &
      3\\
      \hline
      Die Dauer der Dokumentationen wurde unterschätzt &
      3 &
      3\\
      \hline
    \end{tabulary}
    \caption{Risikogewichtung}
  \end{table}
%------------------------------------------------------- 

\subsubsection{Massnahmen}
%------------------------------------------------------- 
%RISK ACTION TABLE
\begin{table}[H]
    \centering
    \settowidth\tymin{executeIncomingCommand()}
    \setlength\extrarowheight{2pt}
    \begin{tabulary}{1.0\textwidth}{|L|L|}
      \hline
      \textbf{Risiko} &
      \textbf{Massnahme}\\
      \hline
      Das Speichern der Projektdaten im Projektordner ist zu wehnig performant &
      Im Rahmen der Diplomarbeit werden keine direkten Massnahmen geplant.
      Performance Optimierungen, respektive Änderungen die daraus folgen werden nach Veröffentlichung der Applikation vorgenommen\\
      \hline
      Krankheitsausfall &
      Bei Eintritt eines krankheitbedingten Ausfalls wird der Diplomlehrer sofort Informiert und abgeklärt ob Anforderungen gekürzt werden können\\
      \hline
      Änderungen durch Unity in Form von Updates, könnten den Einsatz von Custom Editor Windows erschweren oder verunmöglichen &
      Eintrittswarscheinlichkeit während der Projektdauer sehr klein, bei eintritt wird mit der alten Unity-Version weitergearbeitet\\
      \hline
      Nachträgliche Änderungen / Erweiterungen werden verlangt &
      Werden zusätzliche Anforderungen verlangt, so muss der Mehraufwand berechnet werden und deren Folgen auf den Zeitplan.\\
      \hline
      Die Entwicklungsdauer wurde unterschätzt &
      Mit dem Diplomlehrer direkt über mögliche kürzungen der Anforderungen diskutieren\\
      \hline
      Die Dauer der Dokumentationen wurde unterschätzt &
      Mit dem Diplomlehrer direkt über mögliche kürzungen der Anforderungen diskutieren\\
      \hline
    \end{tabulary}
    \caption{Massnahmen}
  \end{table}
%------------------------------------------------------- 
\newpage
\subsection{Wirtschaftlichkeit}
Dadurch, dass die Applikation als Open-Source-Software zu verfügung gestellt wird,
ist eine direkte überprüfung der Wirtschaftlichkeit kaum möglich.\\
Für mich als Entwickler wird die Software nicht Wirtschaftlich sein, 
denn die theoretischen Entwicklungskosten belaufen sich auf rund 20'000 Fr.- (ca. 200h Arbeitsaufwand * 100.- Stundensatz).\\
Die Effektiven Entwicklungskosten sind jedoch eher bei 0 Fr.- anzusiedeln, denn die einzigen Kosten die ich ausser meiner Zeit investiere sind Heiz, Strom und Internetkosten, welche jedoch so oder so anfallen.
\vspace{1cm}\\
Dies bedeutet aber nicht, dass Open-Source-Software nicht Wirtschaftlich ist. Aus Entwickler und Firmensicht stimmt das zwahr,
jedoch können die Nutzer (Mich eingeschlossen) der Software einen erheblichen Wirtschaftlichen Nutzen erhalten. 
Denn die Benutzung der Software ist gratis, es fallen keine zusätzlichen Kosten an und zudem können Entwicklungsdauer und Kosten bei der Entwicklung von Videospielen verkleinert werden.